\documentclass{article}
\usepackage[utf8]{inputenc}
\usepackage{amsmath}
\usepackage{physics}
\usepackage{indentfirst}
\usepackage{qcircuit}
\usepackage{jlcode}
\usepackage[T1]{fontenc}
\usepackage[normalem]{ulem}
\usepackage{listings}
\lstdefinestyle{ascii-tree}{
      literate=%
      {├}{|}1
      {─}{--}1
      {└}{+}1
      {}{\xor}1
      {⟩}{\rangle}1
      {α}{\alpha}1
      {β}{\beta}1
      {⋅}{\cdot}1
}
\lstset{
    language=Julia,
    basicstyle=\ttfamily\footnotesize,
    numberstyle=\scriptsize,
    % numbers=left,
    backgroundcolor=\color{gray!10},
    frame=single,
    tabsize=2,
    rulecolor=\color{black!30},
    title=\lstname,
    escapeinside={\%(*}{*)},
    breaklines=true,
    breakatwhitespace=true,
    framextopmargin=2pt,
    framexbottommargin=2pt,
    extendedchars=true,
    inputencoding=utf8,
    columns=fullflexible,
    mathescape
}
\title{QIP710: Assignment 5}
\author{Xiuzhe Luo (20812697)}
\date{}

\begin{document}

\maketitle

\section*{Some properties of Shor’s 9-qubit code}
\subsection*{1.a}
no, because an counterexample is that if there is a Z-error occurs at 2nd qubit,
the syndrome is the same as occuring on 1st qubit, e.g
if the input state is $\frac{1}{\sqrt{2}}(\ket{0}-\ket{1})$,
both single qubit Z-error on 1st qubit and 2nd qubit give the
same syndrome $00100100$.

* I just check this by a script, the script
is attached in Appendix.

But use the conclusion I prove
in the next question, we can also show that the effect of Z-error in the
same inner block on its syndrome is always to create a phase/or nothing
on the syndrome instead of flip it, so we always get the same bitstring.

\subsection*{1.b}
there can be one X-error occurs in the 3 blocks of the inner
circuit, we pick 2 of the 3 which has 3 possibilities,
then each of them can happen at 3 possible position for each block,
thus $3 \times 3 \times 3 = 27$ weight 2 Pauli-X error in total.

For Z error, it can only occur inside one block, this is because
the inner circuiut leave Z error intact and if there is a Z-error
it always moves it to the first qubit (the protected qubit), when
the ancilla qubits are $\ket{00}$ (note it is not equivalent, I omit
other possible Z gate on 2, 3 wires since they don't change the protected
qubit, so I'm not using \texttt{=}, but $\rightarrow$)

$$
\Qcircuit @C=1em @R=.7em {
    & \ctrl{1} & \ctrl{2} & \multigate{2}{\text{single Z-error}} & \ctrl{2} & \ctrl{1} & \targ & \qw & & 
        & \ctrl{1} & \ctrl{2} & \ctrl{2} & \ctrl{1} & \targ & \gate{Z} & \qw\\
    & \targ & \qw & \ghost{\text{single Z-error}} & \qw & \targ & \ctrl{-1} & \qw & \rightarrow &
        & \targ & \qw & \qw & \targ &  \ctrl{-1} & \qw & \qw\\
    & \qw   & \targ & \ghost{\text{single Z-error}} & \targ & \qw & \ctrl{-1} &\qw & &
        & \qw   & \targ & \targ & \qw & \ctrl{-1} & \qw & \qw
}
$$

this is a statement from the slide, we can prove this easily by applying the swap rule
between Pauli Z and CNOT, note we have

$$
\Qcircuit @C=1em @R=.7em {
    & \gate{Z} & \ctrl{1} & \qw &   & & \ctrl{1} & \gate{Z} & \qw\\
    & \qw      & \targ    & \qw & = & & \targ & \qw & \qw
}
$$

and

$$
\Qcircuit @C=1em @R=.7em {
    & \qw & \ctrl{1} & \qw &   & & \ctrl{1} & \gate{Z} & \qw\\
    & \gate{Z}      & \targ    & \qw & = & & \targ & \gate{Z} & \qw
}
$$

thus after we move the error gates from left to right, we always have one Pauli Z
on the protected qubit.

$$
\Qcircuit @C=1em @R=.7em {
    & \ctrl{1} & \ctrl{2} & \ctrl{2} & \ctrl{1} & \gate{Z} & \targ & \qw\\
    & \targ & \qw  & \qw & \targ & \qw & \ctrl{-1} & \qw \\
    & \qw   & \targ & \targ & \qw & \qw & \ctrl{-1} & \qw
}
$$

$$
\Qcircuit @C=1em @R=.7em {
    & \ctrl{1} & \ctrl{2} & \ctrl{2} & \ctrl{1} & \gate{Z} & \targ & \qw\\
    & \targ & \qw  & \qw & \targ & \gate{Z} & \ctrl{-1} & \qw \\
    & \qw   & \targ & \targ & \qw & \qw & \ctrl{-1} & \qw
}
$$

$$
\Qcircuit @C=1em @R=.7em {
    & \ctrl{1} & \ctrl{2} & \ctrl{2} & \ctrl{1} & \gate{Z} & \targ & \qw\\
    & \targ & \qw  & \qw & \targ & \qw & \ctrl{-1} & \qw \\
    & \qw   & \targ & \targ & \qw & \gate{Z} & \ctrl{-1} & \qw
}
$$

when the ancilla qubits are $\ket{00}$, the Toffoli gate has no effect.
thus if there are two Z error occurs in the inner block, we get two Z gate
on the first qubit, which cancels. But if two Z error occurs at different
inner block, then the outer block has to correct 2 Z-error on 1,4,7, which
is not possible. Thus we can have $3\times 3 = 9$ kinds of weight 2 Z errors.

Regarding to Y-errors, it can only happen when both Z and X error could happen,
which means there cannot be double Y-error since X cannot be in the same block, while
Z has to be in the same block.

Now the ZX-error can happen when single Z-error and single X-error can be corrected, which
is anywhere on the qubits as long as they are not on the same qubit (since we are asking for
weight 2 errors), thus we have $9\times 8=72$ kinds of ZX-error.

ZY-error can only happen in the same block since single X-error can be corrected, but
we cannot have two Z error in different block, thus we have $3\times 6 = 18$ kinds of ZY-error.

XY-error can only happen in different block since single Z-error can be corrected, but
we cannot have two X error in the same block, thus we have $6\times 3 \times 3 = 54$ kinds of XY-error.

Thus we have $27 + 9 + 72 + 18 + 54 = 180$ kinds of weight 2 error.

\subsection*{1.c}
since if Z error occurs at 2,3,5,6,8,9 which cancel itself on each protected qubit, and X
error occurs at 1,4,7 which will be corrected by each inner block, the shor code corrects
a weight 9 error.

\section*{Hadamard transform on uniform superposition of affine linear space}
assume $G$ is the $k\times n$ generator matrix for $x$, for $y \notin C^{\perp}$,
when $Gy \neq 0$

$$
\sum_{x\in C} (-1)^{x\cdot y} = \sum_{k\in \{0, 1\}^k} (-1)^{a G y} = \sum_{k\in \{0, 1\}^k} (-1)^{ab} = 0
$$

since $y \notin C^{\perp}$, G can be viewed as the partity check matrix for $C$, thus $Gy\neq 0$. 
this also indicates that $|C| |C^{\perp}| = 2^n$ So we
have

$$
\begin{aligned}
    H^{\otimes n} (\frac{1}{\sqrt{C}}\sum_{x\in C} \ket{x+z}) &= \frac{1}{\sqrt{C}} \sum_{x\in C} H^{\otimes n}\ket{x+z}\\
    &= \frac{1}{\sqrt{C}} \frac{1}{2^{n/2}} \sum_{x\in C} \sum_{k \in \{0, 1\}^n} (-1)^{x\cdot k} (-1)^{z\cdot k}\ket{k}\\
    &= \frac{1}{\sqrt{|C^{\perp}|}}\sum_{k \in C^{\perp}}(-1)^{z\cdot k} \ket{k} + \frac{1}{\sqrt{C}} \frac{1}{2^{n/2}} \sum_{k \notin C^{\perp}} \sum_{x\in C} (-1)^{x\cdot k}\ket{k}\\
    &= \frac{1}{\sqrt{|C^{\perp}|}}\sum_{k \in C^{\perp}}(-1)^{z\cdot k} \ket{k}
\end{aligned}
$$

\section*{Is the transpose a valid quantum operation?}
\subsubsection*{3.a}
this means $tr(\rho\rho^T) = 0$, $\ket{\psi} = \frac{1}{\sqrt{2}} (\ket{0} + i\ket{1})$ satisfy this condition, the output
pure state is $\frac{1}{\sqrt{2}} (\ket{0} - i\ket{1})$

\subsection*{3.b, c}
quantum channel always maps the density matrix which is a semi-definite matrix to another density matrix which is also a semi-definite
matrix. However, if the transposition is a valid quantum channel, there exists a counterexample, let's apply this channel on the first qubit
of a Bell state $\ket{00} + \ket{11}$:

$$
\Lambda (\ket{00}\bra{00} + \ket{11}\bra{11} + \ket{00}\bra{11} + \ket{11}\bra{00}) = \ket{00}\bra{00} + \ket{11}\bra{11} +
    \ket{10}\bra{01} + \ket{01}\bra{10}
$$

for convenience we omit the normalization factor,
the eigenvalue of the output is not all non-negative (by calculating this numerically,
there is one negative eigenvalue -1), thus it is not a density matrix, thus there is no such channel for all $\rho$.

\section*{Searching with good guessing algorithm}
\subsection*{4.a}
assume we sample $k$ times, the probability of getting a satisfying assignment is

$$
\begin{aligned}
    P(n) &= \sum_{k=1}^n p(1-p)^{k-1} = p \frac{1-(1-p)^n}{p} = 1 - (\frac{3}{4})^n \geq 1-\epsilon\\
    \epsilon &\geq (\frac{3}{4})^n\\
    \ln(\epsilon) &\geq n\ln(\frac{3}{4})\\
    n &\geq \frac{\ln(\epsilon)}{\ln(\frac{3}{4})}
\end{aligned}
$$

\subsection*{4.b}
(note I'm using a different notation of $x$ and $y$ comparing to the question, the $x$ in question is the $k$ here, just for convenience)

denote summation the states satisfy $f$ as $\ket{x} = 2\sum_k A_k \ket{k}, \sum_k |A_k|^2 = 1/4$
and doesn't satisfy $f$: $\ket{y} = \frac{2}{\sqrt{3}}\sum_k B_k \ket{k}, \sum_k |B_k|^2 = 3/4$, we have

$$
\ket{\omega} = U_g \ket{0^n} = \frac{1}{2}\ket{x} + \frac{\sqrt{3}}{2}\ket{y}
$$

since $\ket{x}$ and $\ket{y}$ are on different basis, they are orthogonal, and we have

$$
\begin{aligned}
    U_f \ket{x}\ket{0} &= \ket{x}\ket{1}\\
    U_f \ket{x}\ket{1} &= \ket{x}\ket{0}\\
    U_f \ket{y}\ket{0} &= \ket{y}\ket{0}\\
    U_f \ket{y}\ket{1} &= \ket{y}\ket{1}\\
\end{aligned}
$$

$$
\begin{aligned}
    & U_f (\frac{1}{2} \ket{x}\ket{-} + \frac{\sqrt{3}}{2}\ket{y}\ket{-})\\
    &= -\frac{1}{2}\ket{x}\ket{-} + \frac{\sqrt{3}}{2}\ket{y}\ket{-}
\end{aligned}
$$

Thus $U_f$ reflects the state on $\ket{x}$ axis. Now we create the other reflection
operator $2\ket{\omega}\bra{\omega} - I = U_g (2\ket{0}\bra{0} - I) U_g^{\dag}$, and we
have

$$
\begin{aligned}
    \bra{\omega}\ket{x} &= \frac{1}{2}\\
    \bra{\omega}\ket{y} &= \frac{\sqrt{3}}{2}\\
    (2\ket{\omega}\bra{\omega}-I)\ket{x} &= \ket{\omega} - \ket{x} = -\frac{1}{2}\ket{x} + \frac{\sqrt{3}}{2}\ket{y}\\
    (2\ket{\omega}\bra{\omega}-I)\ket{y} &= \sqrt{3}\ket{\omega} - \ket{y} = \frac{\sqrt{3}}{2}\ket{x} + \frac{1}{2}\ket{y}\\
\end{aligned}
$$

thus apply the reflection after oracle we have

$$
\begin{aligned}
    &(2\ket{\omega}\bra{\omega} - I)(-\frac{1}{2}\ket{x} + \frac{\sqrt{3}}{2}\ket{y})\ket{-}\\
    &=(\frac{1}{4}\ket{x} - \frac{\sqrt{3}}{4}\ket{y} + \frac{3}{4}\ket{x} + \frac{\sqrt{3}}{4}\ket{y})\ket{-}\\
    &=\ket{x}\ket{-}
\end{aligned}
$$

which is the answer! The circuit looks like the following

$$
\Qcircuit @C=1em @R=.7em {
    \ket{0^n} & & \gate{U_g} & \multigate{1}{U_f} & \gate{U_g^{\dag}} & \gate{2\ket{0^n}\bra{0^n} - I} & \gate{U_g} & \qw & \ket{x}\\
    \ket{1} & & \gate{H}   & \ghost{U_f} & \qw & \qw & \qw & \qw & \ket{-}
}
$$

it outputs the answer with only one query without any error.

\section*{Characterizing}
\subsection*{5.a}
assume we have some two qubit state below

$$
\ket{\psi} = A_{1}\ket{00} + A_{2}\ket{01} + A_{3}\ket{10} + A_{4}\ket{11}
$$


on Hadamard basis we have

$$
\begin{aligned}
    2H\otimes H \ket{\psi} =&
        (A_1 + A_2 + A_3 + A_4)\ket{00} +\\
        &(A_1 - A_2 + A_3 - A_4)\ket{01} +\\
        &(A_1 + A_2 - A_3 - A_4)\ket{10} +\\
        &(A_1 - A_2 - A_3 + A_4)\ket{11}
\end{aligned}
$$

in order to have $a=b$ on both basis, we need $A_2 = A_3 = 0$ and $(A_1 - A_2 + A_3 - A_4) = (A_1 + A_2 - A_3 - A_4) = 0$
thus, $A_1 = A_4 = C$, so the state becomes $C (\ket{00} + \ket{11})$

\subsection*{5.b}

similarly we have

$$
\begin{aligned}
    2I\otimes H \ket{\psi} = (A_1 + A_2)\ket{00} + (A_1 - A_2)\ket{01} + (A_3 + A_4)\ket{10} + (A_3 - A_4)\ket{11}\\
    2H\otimes I \ket{\psi} = (A_1 + A_3)\ket{00} + (A_2 + A_4)\ket{01} + (A_1 - A_3)\ket{10} + (A_2 - A_4)\ket{11}
\end{aligned}
$$

thus $A_1 = A_2 = A_3 = C, A_4 = -C$, we get $\frac{1}{2}(\ket{00} + \ket{01} + \ket{10} - \ket{11})$

\newpage
You will need the latest develop version of \texttt{Yao} to run this, which can be installed with \texttt{add Yao\#master}
in Julia's \texttt{pkg} mode.
\begin{lstlisting}
using Yao

shor(Es...) = shor(chain(Es...))

shor(E) = chain(9,
    cnot(1, 4),
    cnot(1, 7),

    put(1=>H), put(4=>H), put(7=>H),

    cnot(1,2), cnot(1,3),
    cnot(4,5), cnot(4,6),
    cnot(7,8), cnot(7,9),
    # error
    E,

    cnot(1,3), cnot(1,2),
    cnot((2, 3), 1),

    cnot(4,6), cnot(4,5),
    cnot((5, 6), 4),

    cnot(7,9), cnot(7,8),
    cnot((8, 9), 7),

    put(1=>H), put(4=>H), put(7=>H),
    cnot(1, 7), cnot(1, 4), cnot((4, 7), 1)
    )

# aasignment answers
# 1.a
@vars α β
r = α * ket"0" + β * ket"1" |> addbits!(8) |> shor(put(1=>Z)) |> partial_tr(1) |> expand
# output: α + β|00100100>
r = α * ket"0" + β * ket"1" |> addbits!(8) |> shor(put(2=>Z)) |> partial_tr(1) |> expand
# α + β|00100100>

# 1.c
r = α * ket"0" + β * ket"1" |> addbits!(8) |>
    shor(kron(1=>X, 2=>Z, 3=>Z, 4=>X, 5=>Z, 6=>Z, 7=>X, 8=>Z, 9=>Z)) |> partial_tr(2:9) |> expand
# output: α|0> + β|1>

# question 5
@vars A1 A2 A3 A4
psi = A1*ket"00" + A2 * ket"01" + A3 * ket"10" + A4 * ket"11"
# 5.a
copy(psi) |> kron(1=>H, 2=>H) |> expand
# 5.b
copy(psi) |> put(2, 1=>H) |> expand
copy(psi) |> put(2, 2=>H) |> expand
\end{lstlisting}


\end{document}